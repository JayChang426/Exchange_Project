\documentclass[12pt]{article}
\usepackage[a4paper, total={15cm,24cm}]{geometry}
\usepackage{titlesec}
\usepackage{graphicx}
\usepackage{float}      % 設定圖片位置
\usepackage{hyperref}
\usepackage{fontspec}   % 加這個就可以設定字體
\usepackage{xeCJK}      % 讓中英文字體分開設置
\usepackage{comment}
\setCJKmainfont{宋體-繁}
\setmainfont{Times New Roman}
\renewcommand{\baselinestretch}{1.5}

\title{The Impact of Tariffs on Price and Welfare: \\ Literature Reviews and Future Research Plans, Presentation Notes}
\author{Ming-Chieh, Chang\thanks{Department of Economics, National Taiwan University}}

\begin{document}

\maketitle


\section*{Page 1}
Today, I'm going to present my report "The Impact of Tariffs on Price and Welfare: Literature Reviews and Future Research Plans". 
I regard this review as a summary of the papers I read on topic of trade and, especially tariffs afetr I was inspired by the Trade Theory class that I took this semester.
And also, I will present my current research plan of my senior thesis, which is largely related to this review.

\section*{Page 2}
So let't start! This is the outline of today's presentation.
Basically I'll give you a short introduction to point out what topics will be covered in this presentation, and then go on to those topics step by step.
At the last part(5) I will discussed how these topics can be applied to Taiwanese trade situation, and why they should be seriously considered.

\section*{Page 3}
OK, These are the topics that will be covered today. I'll start from basic price and welfare effects of tariffs, 
then the distributional effects, which is simply "who win and who lose by a tariff", and an additional topic on productivity effects.
All these topics are well-related to the US-China trade war, so I'm going to apply those conceptual model to this real world event.
Another important reason I use the US-China trade war as the empirical example in this presentation, is that
since Taiwan is an export-oriented economy, the US-China trade war undoubtedly matters to our economy, 

\section*{Page 4}
Alright! This is the graph of the simplest analysis of price and welfare effect of a tariff.
You can see when tariffs are imposed, the export supply curve shifts upward.
Consumer price goes up and price that producer received goes down. 
So the welfare effects are clear, consumers pay rectangle A of the tariff and producer pay rectangle C of the tariff.

\section*{Page 5}
However, this is not the case when the export supply is perfectly elastic.
In this alternative graph, we can see that consumer price goes up, and producers' price remain unchanged.
As a consequence, all tariff burdens fall on consumners, and there's no rectangle C anymore.
In other words, consumers bear all burdens and producers bear no burdens.

\section*{Page 6}
Now we turn to the empirics in the US-Cina trade war.
Unfortunately, the export supply in this event is almost perfectly elastic.
This implies that, importing consumers are paying the tariffs and lose.
We term it as a "complete pass-through" of tariff to consumers.

\section*{Page 7}
So the distributional effects of the trade war is that producers are winners. 
They did not pay tariffs, and they sell products in higher price since those imported before the trade war rose their price.
Losers are definitely consumers.
Especially those in the Midwestern Plains, since they were retaliated the most.

\section*{Page 8}
Interestingly, the finding in differential retaliation degree suggest 
that the most protected counties in the US are those politically competitive counties. 
And this is consistent to the logic of majority voting, 
which suggest a favorable policy for pivotal voters.

This give us some intuition in the "rationale" of the trade war. 
The rationale might be largely connected to political concerns.

\section*{Page 9}
Now let's turn to the "productivity" part.
I'll present two ways to think on this issue.
First is the Melitz model. This is mainly discussion "within an industry".
It said that when firms in a specific industry have diffenent levels of productivity, 
openning to trade can lead higher productivity firms to expand, gain more market share, and increase market efficiency.
So in this case, tariff, which is bad for openness of trade, might block productivity growth.

\section*{Page 10}
Yet, there is also another way to think. 
The Krugman model, which discuss productivity "between" industries, 
said that firms cumulate their productivity through the learning-by-doing process.
So in this case, temporary tariff protection might be beneficial for some infant industries. 
Although they might not be able to compete with foreign companies now, 
they may grow and gain sufficient productivity increase to compete with foreign firms after the tariff protections.
Therefore, tariffs might be good for productivity growth.

\section*{Page 11}
There is no contradiction in the above two ways of thinking.
It is the range of discussion that matters, within or between.
Unfortunately, productivity is more difficult to measure, so literatures on this topic remains small

\section*{Page 12}
So after we talked a lot on trade and tariffs, we need to konw why this issue matters to Taiwan.
Most importantly, Taiwan is very export-oriented, especially to the US and China.
In the short run it might be good for our exporeters to sell products to the US consumers that originnally bought products from China.
However, the global loss of purchasing power may hurt us in the long run.
We don't know what will exactly happaen to Taiwan, so we need econometric methods to measure or even predict it.

\section*{Page 13}
So this is my research plan. 
I want to estimate the short-term gain first.
Then connect those findings to political considerations like waht I mentioned in the rationale part of the US-China trade war.
Hopefully I can provide some suggestions to policymakers based on empirical evidence. 
The productivity part is a topic with great potentials but hard to conduct. 
So I'll put it as an extension and see what I can do for that when I am more skilled with econometric analysis techniques.
That's my presentation, thank you for listening.

\end{document}